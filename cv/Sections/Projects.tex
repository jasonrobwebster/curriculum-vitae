\section{Projects}

\begin{supertabular}{p{1.8cm}|p{12cm}}
	\textsc{Mar 2020} & ASSA Covid Model \\
	& \small\emph{Work Related Project -- Technical Lead}\href{https://github.com/Percept-Health-Solve/seir-model}{\hfill| \footnotesize{Github Link}}\\
	& \footnotesize{Mathematically modelled, built, and maintained the Acturial Society of South Africa's (ASSA's) Covid-19 model. This is an epidemiological model that is built to predict the number of infections, hospitalisations, icu usage, and deaths due to the Covid-19 pandemic in South Africa. The algorithm is built to operate at a national level, as well as a provincial level. This model was a culmination of a team effort of over 20 volunteer actuaries from multiple organisations.} \\
	\multicolumn{2}{c}{} \\
	\textsc{Jan 2019} & Self-driving Car and GamePy \\
	& \small\emph{Personal Project}\href{https://github.com/jasonrobwebster/gamepy}{\hfill| \footnotesize{Github Link}}\\
	& \footnotesize{I built a simulated self-driving car in the game "Burnout: Ultimate Paradise". I gathered data by recording myself playing the game and used this to train a convolutional neural network that mapped images to keystrokes. I managed to make this work using local hardware while being tightly constrained by my budget and GPU size. The result of this project was a general-purpose package dubbed "gamepy" that allows one to record themselves and their keystrokes while acting in a simulated environment. It also allows integration to the game environment to be able to then "play" the game using a prediction of the model. I wrote about my experiences building this in a blog post for OfferZen and can be read \href{https://www.offerzen.com/blog/how-to-develop-a-self-driving-car-in-under-a-week}{here}.} \\
	\multicolumn{2}{c}{} \\
	\textsc{Jun 2018} & AlphaZero Clone \\
	& \small\emph{Personal Project}\href{https://github.com/jasonrobwebster/alphazero-clone}{\hfill| \footnotesize{Github Link}}\\
	& \footnotesize{A clone of the alpha zero algorithm, written entirely by me. The algorithm was used by Google's Deepmind to beat the world's best Go player, as well as the best Chess engine (stockfish). The algorithm learns to play any perfect information board game entirely through self-play and reinforcement learning.} \\
	\multicolumn{2}{c}{} \\
	% \textsc{Mar 2018} & AtomPy \\
	% & \emph{Study Project}\href{https://github.com/jasonrobwebster/atompy}{\hfill| \footnotesize{Github Link}} \\
	% & \footnotesize{I attempted to build a symbolic package for modelling and deriving the equations of motion for an atomic system using Python. I first built the symbolic package from scratch, but later used SymPy as my backend. I managed to get as far as deriving the equations of motion for simple atomic systems, but ended up abandoning the project as it could not handle larger, more complex systems. Note this project is not well documented as it has been abandoned.} \\
	% \multicolumn{2}{c}{} \\
	\textsc{Nov 2017} & Variational Autoencoder \\
	& \small\emph{Personal Project}\href{https://github.com/jasonrobwebster/vae-keras-pytorch}{\hfill| \footnotesize{Github Link}}\\
	& \footnotesize{Variational autoencoders (VAEs) are a neural network architecture that represent high dimensional data (e.g. images) in a low dimensional latent space. This is similar to principle component analysis, however, what makes a VAE different is that it learns the parameters of a probability distribution for each data point, allowing for the generation of new data points. I worked on duplicating this architecture at home, and can now implement a VAE, PixelVAE, and VQ-VAE in tensorflow and keras. The github link shows an example of a simple VAE.} \\
\end{supertabular}